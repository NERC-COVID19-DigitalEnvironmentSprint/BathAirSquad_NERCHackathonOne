\begin{itemize}
\item Is there a correlation between air quality and incidence and severity of COVID-19 infection?
\item What is the air quality threshold we need to improve individual outcomes?
\end{itemize}

is there a positive correlation between air quality (including NOx and particulate pollution levels) and the incidence and severity of COVID-19 infections (based on respiratory stress, a more effective distribution mechanism and greater virus longevity), and if so what is the air quality threshold level we need to reach in order to improve individual outcomes? In a post-lockdown environment, what measures can we put in place to ensure air quality levels stay below these threshold levels (reduced intensity of traffic; greater access to cyclists etc.). How do we manage the tensions between preserving air quality improvements, the use of mass transit systems and the need for social distancing? 