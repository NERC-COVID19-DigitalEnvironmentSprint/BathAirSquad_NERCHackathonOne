\documentclass[11pt, a4paper]{article}
\usepackage[width = 150mm, top = 25mm, bottom = 40mm, bindingoffset = 6mm]{geometry}
\usepackage[utf8]{inputenc}
\usepackage{amsmath}
\usepackage{graphicx}
\usepackage{caption}
\usepackage{amssymb}
\usepackage{wrapfig}
\usepackage{enumitem}
\usepackage{float}
\usepackage{bm}
\usepackage{booktabs}
\usepackage{algorithm}
\usepackage[noend]{algpseudocode}
\usepackage{array}
\usepackage{multirow}
\usepackage{xpatch}
\usepackage{amsthm}
\newtheorem{theorem}{Theorem}
\graphicspath{ {figures/} }

\newcommand{\be}{\begin{equation}}
\newcommand{\ee}{\end{equation}}
\newcommand{\bea}{\begin{equation*}}
\newcommand{\eea}{\end{equation*}}
\newcommand{\Prob}{\mathbb{P}}
\newcommand{\E}{\mathbb{E}}
\newcommand{\R}{\mathbb{R}}
\newcommand{\N}{\mathbb{N}}
\newcommand{\Z}{\mathbb{Z}}
\newcommand{\V}{\mathbb{V}}
\newcommand{\p}{\partial}
\newcommand{\im}{\text{i}}
\newcommand{\infint}{\int_{-\infty}^\infty}
\newcommand{\F}{\mathcal{F}}

\numberwithin{equation}{section}

\title{NERC Air Quality Hackathon Notes: UK ideas}
\author{Dan Burrows}
%\date{May 2020}
\begin{document}

\maketitle

\section{Problem outline}
\begin{itemize}
\item Is there a correlation between air quality and incidence and severity of COVID-19 infection?
\item What is the air quality threshold we need to improve individual outcomes?
\end{itemize}
Is there a positive correlation between air quality (including NOx and particulate pollution levels) and the incidence and severity of COVID-19 infections (based on respiratory stress, a more effective distribution mechanism and greater virus longevity), and if so what is the air quality threshold level we need to reach in order to improve individual outcomes? In a post-lockdown environment, what measures can we put in place to ensure air quality levels stay below these threshold levels (reduced intensity of traffic; greater access to cyclists etc.). How do we manage the tensions between preserving air quality improvements, the use of mass transit systems and the need for social distancing? 

\section{Data}
\begin{itemize}
\item Collate the lab cases and deaths for UK cities.
\item Need to consider the effect of the variates for each of these cities e.g. for each individual (or on average) we must consider the relative
\begin{enumerate}
\item Wealth
\item Lifestyle
\item Profession
\item Life expectancy
\item Area mortality rate
\item Hospital mortality rate (some hospitals may be more stretched than others)
\end{enumerate}
\end{itemize}
If we can somehow isolate the effect of air quality then we can quantify any positive correlation between the air quality and COVID-19 incidence/severity.

\section{Model}
Quantification of COVID-19:
\begin{itemize}
\item In line with the problem outline our response variables should be "respiratory stress, a more effective distribution mechanism (what is this?) and greater virus longevity".
\item Incidence could be quantified as the R number, given the air quality level.
\end{itemize}

Piotr is thinking about a model for the new cases.



















\end{document}